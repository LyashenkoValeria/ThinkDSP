\documentclass[a4paper,12pt]{report}

%Русский язык
\usepackage[T2A]{fontenc}
\usepackage[utf8]{inputenc}
\usepackage[english,russian]{babel}
\usepackage{cmap}

%Работа с кодом
\usepackage{listings}
\usepackage{color}

\definecolor{green}{rgb}{0,0.6,0}
\definecolor{gray}{rgb}{0.5,0.5,0.5}
\definecolor{red}{rgb}{0.6,0,0}

\lstset{
        language=Python, 
        basicstyle=\small\ttfamily, 
        numberstyle=\tiny,           
        columns=flexible,
        stepnumber=1,                   
        numbersep=5pt,        
        showspaces=false,
        showstringspaces=false,
        showtabs=false,
        tabsize=2,                
        captionpos=b,              
        breaklines=true,           
        breakatwhitespace=false,
        keywordstyle=\color{green},
        commentstyle=\color{gray},
        stringstyle=\color{red},      
}

%Математика
\usepackage{amsmath,amsfonts,amssymb,amsthm,mathtools} 

%Изображения
\usepackage{float}
\usepackage{graphicx}
\graphicspath{ {./img/} }

%Поля страницы
\usepackage{geometry} 
\geometry{left=2.3cm} 
\geometry{right=1.8cm} 
\geometry{top=2cm} 
\geometry{bottom=2.5cm} 

%Отступы
\usepackage{indentfirst}
\setlength{\parskip}{0cm}

\begin{document} 

\begin{titlepage}
\newpage
	\begin{center}
		\large Санкт-Петербургский политехнический университет Петра Великого\\
		Институт компьютерных наук и технологий\\
		Высшая школа интеллектуальных систем и суперкомпьютерных технологий\\
	\end{center}
\vspace{7cm}

\begin{center}
		\large \textbf{Отчёт по лабораторной работе №7} \\
		\textbf{Дисциплина:} Телекоммуникационные технологии\\
		\textbf{Тема:} Дискретное преобразование Фурье
\end{center}
\vspace{4cm}
	
\begin{flushright}
		\large Работу выполнил:\\ Ляшенко В.В.\\
		Группа: 3530901/80201\\
		Преподаватель:\\ Богач Н.В.
\end{flushright}

\vspace{\fill}
\begin{center}
	\large Санкт-Петербург\\ 2021
	\end{center}
\end{titlepage}

\tableofcontents
\listoffigures
\lstlistoflistings

\chapter{Упражнение 7.1}
    В начале мы должны для Jupyter загрузить \texttt{chap07.ipynb}, прочитать пояснения и запустить примеры.
    
    Все примеры были успешно запущены.

\chapter{Упражнение 7.2} 
\section{Алгоритм}
    Реализуем алгоритм быстрого преобразования Фурье (БПФ), время работы которго $N \log N$. Для этого воспользуемся леммой Дэниелсона-Ланцоша. 
\begin{equation}
       DFT(y)[n] = DFT(e)[n]+exp(-2nin/N)DFT(o)[n]
\end{equation}

    В этой формуле DFT($y$)[n] - это $n$-й элемент ДПФ от $y$, $e$ и $o$ - массивы сигнала, содержащие соответственно четные и нечетные элементы $y$.
    
    Эта лемма предлагает рекурсивный алгоритм для ДПФ:
\begin{enumerate} 
  \item Дан массив сигнала $y$. Разделим его на чётные элементы $e$ и нечётные элементы $o$. 
  \item Вычислим DFT $e$ и $o$, делая рекурсивные вызовы.
  \item Вычислим DFT($y$) для каждого значения $n$, используя лемму Дэниелсона-Ланцоша. 
\end{enumerate}

    В простейшем случае эту рекурсию надо продолжать, пока длина $y$ не дойдет до 1. Тогда DFT($y$) = $y$. А если длина $y$ достаточно мала, можно вычислить его ДПФ перемножением матриц, используя заранее вычисленные матрицы.
    
\section{Реализация}
    Возьмём небольшой сигнал и вычислим его БПФ с помощью имеющейся функции \texttt{np.fft.fft}.
\begin{lstlisting}[caption=Вычисление БПФ с помощью np.fft.fft]
       ys = [-0.3, 0.8, 0.6, -0.4]
       hs = np.fft.fft(ys)
       print(hs)
\end{lstlisting}  
\begin{lstlisting}[caption=Полученные результаты]
       [ 0.7+0.j  -0.9-1.2j -0.1+0.j  -0.9+1.2j]
\end{lstlisting}

    Теперь реализуем функцию ДПФ.
\begin{lstlisting}[caption=Функция dft]
       def dft(ys):
           N = len(ys)
           ts = np.arange(N) / N
           freqs = np.arange(N)
           args = np.outer(ts, freqs)
           M = np.exp(1j * PI2 * args)
           amps = M.conj().transpose().dot(ys)
           return amps
\end{lstlisting} 
    
    Воспользуемся ей, чтобы убедиться, что результаты одинаковые. 
\begin{lstlisting}[caption=Применение функции ДПФ]
       hs2 = dft(ys)
       np.sum(np.abs(hs - hs2))
\end{lstlisting}  
\begin{lstlisting}[caption=Полученные результаты]
       1.0537365376317067e-15
\end{lstlisting}

    Как мы можем видеть, различия минимальны.
    
    Для того, чтобы создать рекурсивное БПФ, напишем функцию, которая разбивает входной массив и использует np.fft.fft для вычисления БПФ полученных половин.
\begin{lstlisting}[caption=Функция fft\_norec]
       def fft_norec(ys):
           N = len(ys)
           He = np.fft.fft(ys[::2])
           Ho = np.fft.fft(ys[1::2])
    
           ns = np.arange(N)
           W = np.exp(-1j * PI2 * ns / N)
    
           return np.tile(He, 2) + W * np.tile(Ho, 2)
\end{lstlisting}  
    
    Применим эту функцию и убедимся, что результат тот же.
\begin{lstlisting}[caption=Применение функции fft\_norec]
       hs3 = fft_norec(ys)
       np.sum(np.abs(hs - hs3))
\end{lstlisting}  
\begin{lstlisting}[caption=Полученные результаты]
       3.820527793534411e-16
\end{lstlisting} 

    Разница также мала.
    
    Теперь реализуем функцию \texttt{fft}, где заменим np.fft.fft на рекурсивные вызовы.
\begin{lstlisting}[caption=Функция fft]
       def fft(ys):
           N = len(ys)
           if N == 1:
               return ys
    
           He = fft(ys[::2])
           Ho = fft(ys[1::2])
    
           ns = np.arange(N)
           W = np.exp(-1j * PI2 * ns / N)
    
           return np.tile(He, 2) + W * np.tile(Ho, 2)
\end{lstlisting}

    Проверим её работу.
\begin{lstlisting}[caption=Применение функции fft]
       hs4 = fft(ys)
       np.sum(np.abs(hs - hs4))
\end{lstlisting}  
\begin{lstlisting}[caption=Полученные результаты]
       3.820527793534411e-16
\end{lstlisting} 

    Всё работает верно.
    
    Полученная реализация работает за $N \log N$, однако имеют недостатки - занимаемое пространство так же составляет $N \log N$, к тому же тратится время на создание и копирование массивов. 
    
    Функцию можно улучшить, реализовав работу «на месте».
\chapter{Выводы}
    В результате выполнения данной работы мы изучили дискретное преобразование Фурье и быстрое преобразование Фурье.
    Во многих случаях достаточно ДПФ, которое работает за $N^2$, но иногда требуется БПФ, которое работает за $N \log N$.
\end{document}